\documentclass[a4,12Pt]{article}
\usepackage[utf8]{inputenc}


\begin{document}
%\maketittle
\begin{flushright}
Artis Erglis\\
x141REB143\\
\end{flushright}

\begin{center}
\large LD22H
\end{center}

\begin{flushleft}
1)Darba izpildes sakums - 5. februaris\\
2)Darba pabeigšanas - 11. februaris
\end{flushleft}

\begin{verbatim}
1)https://213.175.92.37/~x141REB143/221.svg
2)https://213.175.92.37/~x141REB143/222.svg
3)https://213.175.92.37/~x141REB143/223.svg
4)https://213.175.92.37/~x141REB143/224.svg
5)https://213.175.92.37/~x141REB143/225.svg
6)https://213.175.92.37/~x141REB143/226.svg
7)https://213.175.92.37/~x141REB143/227.svg
8)https://213.175.92.37/~x141REB143/228.svg
9)https://213.175.92.37/~x141REB143/darbi2/221.sch
10)https://213.175.92.37/~x141REB143/darbi2/221.py
11)https://213.175.92.37/~x141REB143/darbi2/222.py

\end{verbatim}

\section{Secinajumi}
\begin{flushleft}
Veicot labaratorijas darbu apguvu jaunas iemannas, kuras man turpmaak noderees. Intersantaakais laboratorijas darba punkts bija savas web lapas veidoshana.
Secinaju , ka vektor grafika ir labaaka par rastrgrafiku.
\end{flushleft}
\end{document}
